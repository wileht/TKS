\documentclass[12pt,a4paper,leqno]{report}

\usepackage[UTF8]{inputenc}
\usepackage[T1]{fontenc}
\usepackage[finnish]{babel}
\usepackage{amsthm}
\usepackage{amsfonts}         
\usepackage{amsmath}
\usepackage{amssymb}
\usepackage{graphicx}
\usepackage{array}
\usepackage{hyperref}
\usepackage{rotating}

\pagestyle{plain}
\setcounter{page}{1}
\addtolength{\hoffset}{-1.15cm}
\addtolength{\textwidth}{2.3cm}
\addtolength{\voffset}{0.45cm}
\addtolength{\textheight}{-0.9cm}

\title{Tietokantasovellus: Keskustelufoorumi (dokumentaatio)}
\author{Wille Lehtomäki}
\date{}

\begin{document}

\maketitle

\tableofcontents

\chapter{Johdanto}\label{johd}

Työssä toteutetaan PHP-kielellä keskustelufoorumi, jonka käyttämiseen (siis pelkkään lukemiseenkin) tarvitaan käyttäjätunnus. Foorumilla on yksi ylläpitäjä, joka voi tavallisen käyttäjän oikeuksien lisäksi myös esimerkiksi poistaa viestejä ja lisätä viesteille aihekategorioita.

Foorumi käyttää PostgreSQL-tietokantaa.

\chapter{Yleiskuva järjestelmästä}

\begin{center}
\includegraphics[scale=0.5]{kayttp}
\end{center}

\noindent \textbf{Huom:} Ylläpitäjä voi tehdä kaikki samat asiat kuin Käyttäjäkin, mutta luettavuuden vuoksi nämä yhteydet on jätetty kaaviossa merkitsemättä.

\Large\textbf{Käyttäjäryhmät}\normalsize\\

\emph{Käyttäjä} on foorumin ns. "tavallinen"  käyttäjä, eli henkilö, joka voi esimerkiksi lukea ja kirjoittaa viestejä.\\

Foorumin \emph{ylläpitäjällä} on lisäksi oikeus mm. poistaa viestejä ja määritellä viesteille aihealueita niiden järjestelemiseksi sekä hallita käyttäjäryhmiä.\\ \\

\Large\textbf{Käyttötapauskuvauksia}\normalsize\\

\textbf{Viestin lisääminen}

\noindent Käyttäjä kirjoittaa ja lisää foorumille uuden aloitusviestin, eli viestin, joka ei ole vastine mihinkään vanhaan viestiin. Muut käyttäjät voivat sekä lukea viestin että kirjoittaa sille vastineita.\\

\textbf{Vastineen lisääminen}

\noindent Vastine on vastausviesti johonkin aiemmin kirjoitettuun viestiin.\\

\textbf{Viestin lukeneiden näkeminen}

\noindent Käyttäjälle näytetään lista kaikista käyttäjistä, jotka ovat lukeneet tietyn viestin.\\

\textbf{Viestien hakeminen}

\noindent Viestejä voi hakea kirjoittajan, aiheen tai kirjoituspäivän perusteella.\\

\textbf{Viestin poistaminen}

\noindent Ylläpitäjä voi poistaa joko yksittäisiä vastineita tai kokonaisen viestiketjun poistamalla ketjun aloitusviestin.\\

\textbf{Käyttäjäryhmän luominen}

\noindent Ylläpitäjä voi luoda käyttäjille ryhmiä.\\

\textbf{Aiheen lisäys}

\noindent Käyttäjien ryhmittelyn lisäksi ylläpitäjä voi ryhmitellä myös viestejä asettamalla viesteille eri aihealueita, joihin kukin viesti mahdollisine vastineineen kuuluu.\\

\noindent Loput käyttötapaukset on listattu käyttötapauskaaviossa.

\chapter{Järjestelmän tietosisältö}

\begin{center}
\includegraphics[scale=0.55]{tietosisaltokaavio}
\end{center}

\renewcommand{\arraystretch}{1.5}

\noindent \textbf{Käyttäjäryhmä}

\begin{center}
\begin{tabular}{| l | l | l | l |}
\hline
\textbf{Attribuutti} & \textbf{Arvojoukko} & \textbf{Kuvaus} \\ \hline
Nimi & Merkkijono, max 20 merkkiä & Käyttäjäryhmän nimi \\ \hline
\end{tabular}
\end{center}
\ \\

\noindent \textbf{Käyttäjä}

\begin{center}
\begin{tabular}{| l | l | l | l |}
\hline
\textbf{Attribuutti} & \textbf{Arvojoukko} & \textbf{Kuvaus} \\ \hline
Käyttäjätunnus & Merkkijono, max 20 merkkiä & Käyttäjän käyttäjätunnus \\ \hline
Salasana & Merkkijono, max 20 merkkiä & Käyttäjän salasana (selkokielinen) \\ \hline
\end{tabular}
\end{center}
\ \\

 \noindent \textbf{Aloitusviesti}

\begin{center}
\begin{tabular}{| l | l |m{0.3\linewidth}|}
\hline
\textbf{Attribuutti} & \textbf{Arvojoukko} & \textbf{Kuvaus} \\ \hline
Tunnus & Kokonaisluku & Viestin yksilöivä tunnus \\ \hline
Kirjoittaja & Merkkijono, max 20 merkkiä & Viestin kirjoittaneen käyttäjän käyttäjätunnus \\ \hline
Keskustelualue & Merkkijono, max 50 merkkiä & Mihin keskustelualueeseen viesti kuuluu \\ \hline
\end{tabular}
\end{center}
\ \\

\noindent \textbf{Vastine}

\begin{center}
\begin{tabular}{| l | l |m{0.3\linewidth}|}
\hline
\textbf{Attribuutti} & \textbf{Arvojoukko} & \textbf{Kuvaus} \\ \hline
Tunnus & Kokonaisluku & Vastineen yksilöivä tunnus \\ \hline
Kirjoittaja & Merkkijono, max 20 merkkiä & Viestin kirjoittaneen käyttäjän käyttäjätunnus \\ \hline
Aloitusviesti & Kokonaisluku & Sen viestin tunnus, johon vastine on vastaus \\ \hline
\end{tabular}
\end{center}
\ \\

\noindent \textbf{Keskustelualue}

\begin{center}
\begin{tabular}{| l | l | l | l |}
\hline
\textbf{Attribuutti} & \textbf{Arvojoukko} & \textbf{Kuvaus} \\ \hline
Nimi & Merkkijono, max 50 merkkiä & Keskustelualueen nimi \\ \hline
\end{tabular}
\end{center}

\chapter{Relaatiotietokantakaavio}

\begin{center}
\includegraphics[scale=0.50]{relaatiotietokantakaavio}
\end{center}

\chapter{Järjestelmän yleisrakenne}

Työ noudattaa MVC-mallia, mutta kaikki kontrollerit (apukontrollereineen) on sijoitettu projektin juureen. Yleiskäyttöiset, lähinnä istuntoa koskevat apufunktiot on sijoitettu libs-kansion funtkiot.php-tiedostoon.

Näkymät puolestaan ovat views-kansiossa, ja mallit libs-kansion alikansiossa models (malliluokkien tiedostonimet on kirjoitettu isolla alkukirjaimella).

Vain kirjautumis- ja rekisteröitymissivuja on mahdollista tarkastella kirjautumatta sisään. Ylläpitoalueen sivut puolestaan ovat vain ylläpitäjien nähtävissä.

\chapter{Käyttöliittymä ja järjestelmän komponentit}

Sivukartta (seuraavalla sivulla):

\begin{sideways}
\includegraphics[scale=.55]{sivukartta}
\end{sideways}

\chapter{Asennustiedot}

Tulossa.

\chapter{Käynnistys- / käyttöohje}

Normaalin käyttäjän tunnus ja salasana: Tapsa, koppelo\\

\noindent Tällä hetkellä foorumille voi lähettää uusia aloitusviestejä, ja niiden kirjoittaja ja ylläpitäjät voivat myös muokata niitä. Kaikki käyttäjät voivat vastata aloitusviesteihin, ja vastineita voi muokata ja poistaa.\\

\noindent Foorumille on nyt myös mahdollista tehdä uusi käyttäjätunnus.\\

\noindent Linkkejä:\\
\href{http://wlehtoma.users.cs.helsinki.fi/Keskustelufoorumi/esittelysivu.html}{Esittelysivu}\\
\href{http://wlehtoma.users.cs.helsinki.fi/Keskustelufoorumi/html-demo/}{Html-demo}\\
\href{http://wlehtoma.users.cs.helsinki.fi/Keskustelufoorumi/kirjautuminen.php}{Foorumi}

\chapter{Testaus, tunnetut bugit ja puutteet \& jatkokehitysideat}

Ylläpitoalue ei tällä hetkellä ole toiminnallinen.

\chapter{Omat kokemukset}

Tulee kun työ on valmis.

\end{document}